\documentclass{article}
\usepackage[spanish]{babel}
\usepackage[utf8]{inputenc}
\usepackage{geometry}
\usepackage{tikz}
\usepackage{pgfplots}
\usepackage{cancel}
\usepackage{amsmath, amsthm, amssymb}
\usepackage{color}
%\usepackage{multicol}
%\usepackage{amsfonts}

\pgfplotsset{compat=1.18}

\newcommand{\RR}{\mathbb{R}}
\newcommand{\NN}{\mathbb{N}}
\newcommand{\QQ}{\mathbb{Q}}
\newcommand{\II}{\mathbb{I}}
\newcommand{\ZZ}{\mathbb{Z}}
\newcommand{\PP}{\mathbb{P}}

\begin{document}
\title{Gr\'afico de polinomios cuadr\'aticos (par\'abolas)}
\author{Ramiro Rebolledo}
\date{\today}
\maketitle

\section{Objetivos de la unidad}
\begin{itemize}
\item Dibujar gr\'aficos de polinomios cuadr\'aticos (par\'abolas)
  en el plano cartesiano.
\item Calcular y comprender gr\'aficamente las intersecciones del polinomio
  cuadr\'atico con el eje $X$ (sus ra\'ices o ceros).
\item Determinar gr\'aficamente los intervalos donde el polinomio
  cuadr\'atico es positivo y los intervalos donde es negativo.
\item Decidir a partir de la ecuaci\'on cuadr\'atica cuando su gr\'afica es
  c\'oncava hacia arriba y cuando es c\'oncaba hacia abajo.
\end{itemize}

\section{Definici\'on de polinomio cuadr\'atico}
Un polinomio cuadr\'atico se define como
\[
  p(x) = \textcolor{red}{a}x^2 + bx + c,
\]
donde $a$, $b$, y $c$ son constantes con $\textcolor{red}{a \neq 0}$. 
Cuando $a = 0$, la ecuaci\'on corresponde a un polinomio lineal.

\begin{itemize}
\item $a$, $b$ y $c$ se denominan \textbf{coeficientes} del polinomio.
\item Los coeficientes del polinomio son \textbf{independientes}
  de la variable.
\item $\textcolor{red}{a}$ se denomina \textbf{coeficiente principal},
  y determina el \textbf{grado} del polinomio.
\item $c$ se denomina \textbf{t\'ermino libre}.
\end{itemize}

\noindent
\textbf{Observaci\'on:}
Cuando $a=0$, la ecuaci\'on corresponde a un polinomio lineal
(polinomio de primer grado).
Por este motivo, excluimos este caso del an\'alisis.

\medskip

\noindent
\textbf{Ejemplos:}
Ejemplos de polinomios de segundo grado.
\begin{itemize}
\item $p(x)=ax^2+bx+c$, donde $a$, $b$ son constantes y $a\neq 0$.
\item $p(x)=2x^2+x+1$
\item $p(x)=1+x+2x^2$
\item $f(x)=0.7x^2-12.3x+\dfrac{3}{2}$
\item $p(w)=-3w^2$
\item $p(x)=-x-3x^2$
\item $p(t)=2t^2+t+1$
\item $p(t)=3+5t-7t^2$
\item $y(t)=y_0+v_{0y}t+\dfrac{1}{2}gt^2$, donde $y_0$, $v_{0y}$ y $g$
  son constantes.
\item $p(x)=x^2+\dfrac{2}{x}+1$ \textbf{no es un polinomio}.
\item $p(x)=\dfrac{1}{x^2+2x+1}$ \textbf{no es un polinomio}.
\end{itemize}



\section{Gr\'afico de polinomios cuadr\'aticos}

Analizaremos el gr\'afico del polinomio
\[
  p(x)=\textcolor{red}{a}x^2+bx+c\,,
\]
donde \textcolor{red}{$a\neq 0$}.

Sabemos que
\[
  \boxed{ax^2+bx+c=0}
  \qquad\Leftrightarrow\qquad
  \boxed{x=\dfrac{-b\pm\sqrt{\textcolor{brown}{b^2-4ac}}}{2a}}
\]
lo que gr\'aficamente quiere decir que, cuando
\textcolor{brown}{$b^2-4ac\geq 0$},
el polinomio interseca al eje $X$ en los puntos con abcisa
\[
  x_1=\dfrac{-b\textcolor{red}{-}\sqrt{b^2-4ac}}{2a}
  \quad\text{ y }\quad
  x_2=\dfrac{-b\textcolor{red}{+}\sqrt{b^2-4ac}}{2a}
\]





    

\subsection{Ceros o ra\'ices de polinomios cuadr\'aticos}
Las ra\'ices de un polinomio cuadr\'atico $p(x) = ax^2 + bx + c$, con
$a\neq 0$, son
\[x_1 = \frac{-b - \sqrt{b^2 - 4ac}}{2a}\]
y
\[x_2 = \frac{-b + \sqrt{b^2 - 4ac}}{2a}.\]

\subsection{Concavidad}
La concavidad de la gr\'afica de un polinomio cuadr\'atico depende del signo
de $a$. Si $a > 0$, la par\'abola es c\'oncava hacia arriba, y si $a < 0$,
la par\'abola es c\'oncava hacia abajo.
M\'as informaci\'on, puede revisar el libro \cite{Z08}.


\bibliographystyle{plain}
\bibliography{bibliografia}


\end{document}

