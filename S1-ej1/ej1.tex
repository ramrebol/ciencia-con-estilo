\documentclass[11pt]{article}
\usepackage[spanish]{babel}
\usepackage{geometry}
\geometry{
  letterpaper,
  total={216mm,279mm},
  left=20mm,
  right=20mm,
  top=20mm,
  bottom=40mm
}

\usepackage{amsmath,amsfonts,amssymb}
\usepackage{multicol}
\usepackage{xcolor}

% \everymath{\displaystyle}
\setlength\parindent{0pt}

\newcommand{\ds}{\displaystyle}

\newcommand{\RR}{\mathbb{R}}
\newcommand{\ZZ}{\mathbb{Z}}
\newcommand{\NN}{\mathbb{N}}

\begin{document}
%%%%%%%%%%%%%%%%%%%%%%%%%%%%%%%%%%%%%%%%%%%%%%%%%%%%%%%%%%%%%%%%%%%%%%%%%%%%%%%%
UNIVERSIDAD DE CONCEPCI\'ON
\\
\textbf{\small FACULTAD DE INGENIER\'IA AGR\'ICOLA}
\\
DEPARTAMENTO DE AGROINDUSTRIAS\\
\rule{16cm}{.5pt}
%  \renewcommand{\theenumi}{\Roman{enumi}}
% \renewcommand{\theenumii}{\arabic{enumi}.\arabic{enumii}}
% 
% \renewcommand{\theenumiii}{\arabic{enumi}.\arabic{enumii}.\arabic{enumiii}}

\vspace{0.5cm}

\begin{center}
  \textbf{Listado 5: Funciones exponencial y logaritmo}
  \\
  \textsc{\'Algebra y Trigonometr\'ia}
\end{center}
%%%%%%%%%%%%%%%%%%%%%%%%%%%%%%%%%%%%%%%%%%%%%%%%%%%%%%%%%%%%%%%%%%%%%%%%%%%%%%%%

\begin{enumerate}
\item Escribir la expresi\'on indicada en una forma logar\'itmica equivalente:
  \begin{enumerate}
    \begin{multicols}{4}
    \item $9^{-1/2}=1/3$
    \item \textcolor{red}{HACER}
    \item $4^0=1$
    \item \textcolor{red}{HACER}
    \item $5^y=x$
    \item $\sqrt[x]{2}=3$
    \item $e^{2u}=3v$
    \item $y=e^x$
    \end{multicols}
    
  \end{enumerate}

  % ----------------------------------------------------------------------------
  
\item Escribir la expresi\'on indicada en una forma exponencial equivalente:


  % ----------------------------------------------------------------------------

\item Sin utilizar calculadora, encontrar el valor de los siguientes logaritmos:
  \begin{enumerate}
    \begin{multicols}{3}
    \item $\log_{10} 0.00001$
    \item $\log_21/64$
    \item \textcolor{red}{HACER}
    \item $\log_{\sqrt{3}}9$
    \item $\log_81/16$
    \item \textcolor{red}{HACER}
    \end{multicols}	
  \end{enumerate}

  % ----------------------------------------------------------------------------
  
\item En cada caso, despejar la inc\'ognita:
  \begin{enumerate}
    \begin{multicols}{4}
    \item $\log_a64=3$
    \item \textcolor{red}{HACER}
    \item $\log_71/x=-3$
    \item $\log_636^c=18$
    \item \textcolor{red}{HACER}
    \item $\log_b8=-1$
    \item $\log_28^y=6$
    \item \textcolor{red}{HACER}
    \end{multicols}
  \end{enumerate}

  % ----------------------------------------------------------------------------

\item Para cada una de las siguientes expresiones, simplificar y reducir a un
  solo logaritmo:
  \begin{enumerate}
    \begin{multicols}{2}
    \item $\log_63+\log_62$
    \item $\dfrac{1}{3}\log_364-\dfrac{1}{2}\log_325+20\log_31$
    \item $\ln(x^4-1)-\ln(x^2+1)$
    \item $\ln\left( \dfrac{a}{b}\right)
      -3\ln\left( a^2\right) +\ln\left( b^{-2}\right) $		
    \end{multicols}	
  \end{enumerate}

  % ----------------------------------------------------------------------------
  
\item Resolver las siguientes ecuaciones exponenciales con $x\in\RR$:
  \begin{enumerate}
    \begin{multicols}{3}
    \item $\left(\dfrac{3}{4}\right)^{2x}\left(\dfrac{8}{3}\right)^{2x}=2^{x-3}$
    \item \textcolor{red}{HACER}
    \item $3^{x^2-5}=81$
    \item $4^x=\sqrt{32}$
    \item $10^{2x-1}-10^x=0$
    \item $2\cdot 3^{x-1}+3^{2x}=\dfrac{1}{3}$
    \item $2\cdot 9^x=3^x+1$
    \item $9^x-2\cdot 3^x=8$
    \item \textcolor{red}{HACER}
    \item $5^{2x}-6\cdot 5^x+5=0$
    \item $4^x-2^x=2$
    \item $4\cdot 5^{2x}-4\cdot 5^x+1=0$
    \item $25^x-6\cdot 5^x+5=0$
    \item ${2^{x+2}+2^{x+3}+2^{x+4}+2^{x+5}+2^{x+6}=31}$
    \end{multicols}
  \end{enumerate}

\newpage
  
  % ----------------------------------------------------------------------------

\item Resolver las siguientes ecuaciones logar\'itmicas con $x\in\RR$:
  \begin{enumerate}
    \begin{multicols}{2}
    \item $\log_{10} x^5+\log_{10}^2x+6=0$
    \item $\log_2x\left( \log_2x+1\right) =2$
    \item $\log_{10} x+\log_{10}(x+3)=1$
    \item $\log_{10} \left(7x-9\right)^2+\log_{10} \left(3x-4\right)^2=2$
    \item $\ln\vert x+1\vert+\ln \vert x+3\vert =\ln 8$
    \item $\sqrt[x]{2}=3$
    \item $\log_x10-\log_{10} x^2=1$
    \item $\log_2\left( 9^{x-1}+7\right) =2+\log_2\left( 3^{x-1}+1\right) $
    \item $(\dfrac{1}{25})^{x-2}\leq \left(\dfrac{1}{5}\right)^{-2x+8}$
      \textcolor{red}{CORREGIR}
    \item $1<\sqrt[x-1]{32^{2x+5}}$
    \item $e^{3x}+2e^{2x}-8e^x<0$
    \item $\log_3(x^2-3x-4)<1$
    \item $\log_2(\log_{\frac{1}{2}}(x+1))>1$
    \end{multicols}
  \end{enumerate}

  % ----------------------------------------------------------------------------
  
\item Determinar el dominio de las siguientes funciones reales:
  \begin{enumerate}
    \begin{multicols}{2}
    \item $f(x)=4^x+2$
    \item $f(x)=\left(1/2\right)^x$
    \item $f(x)=3^{2-x}$
    \item $f(x)=\ln (x^2-1)$
    \item $f(x)=\ln (8-x^3)$
    \item $f(x)=\sqrt{\log_a(x+1)}$, $0<a<1$
    \item $f(x)=\ln (x^2-9)$
    \item $f(x)=\ln (\vert \vert x\vert-1\vert- 1)$
    \item \textcolor{red}{HACER}
    \item $f(x)=e^{\ln x}$
    \item $f(x)=\ln (e^x)$
    \end{multicols}
  \end{enumerate}

  % ----------------------------------------------------------------------------
  
\item Para las funciones del problema anterior, decidir si existe la funci\'on
  inversa, en caso negativo, hacer las restricciones que sean necesarias de
  modo de ella exista y definir dicha inversa; adem\'as, para las tres
  primeras, esbozar los gr\'aficos de $f$ y $f^{-1}$.

  % ----------------------------------------------------------------------------
  
\item Se dice que una variable $A$ posee un crecimiento exponencial,
  si $A(t) = A_0e^{k\,t}$, donde $A_0$ es el valor inicial y $k>0$ es una
  constante. Si un cierto tipo de bacteria tiene un crecimiento exponencial
  e inicialmente hay $100$ bacterias presentes y cinco horas m\'as tarde hay
  $300$ bacterias presentes. ?`Cu\'antas habr\'a diez horas despu\'es del
  inicio? \textbf{R:} 900 bacterias

  % ----------------------------------------------------------------------------

\item La Ley de enfriamiento de Newton establece que la temperatura $T$ de un
  cuerpo enfri\'andose est\'a dada por $T(t)=Ce^{-kt}+T^*$, donde $T^*$ es la
  temperatura del medio que lo rodea, con $C$ y $k$ constantes. Usar esta ley
  para resolver el siguiente problema:

  Se ha cometido un asesinato. El cad\'aver fue encontrado a las 13:00 horas,
  momento en que se le tom\'o la temperatura, siendo de $32^\circ C$.
  Dos horas m\'as tarde la temperatura fue nuevamente registrada,
  siendo de $29^\circ C$. La temperatura del lugar del lugar donde se
  encontraba el cuerpo era constante y de $28^\circ C$.
  Si la temperatura de un cuerpo en \textit{estado normal} es de $36^\circ C$,
  ?`a qu\'e hora se cometi\'o el asesinato?
  \begin{enumerate}
  \item Considerando que $T(0)=32$, determinar el valor de la constante $C$.
  \item Con el valor de $C$ antes obtenido y considerando que $T(2)=29$,
    determinar el valor de la constante $k$.
  \item Con los valores de $C$ y $k$ ya conocidos, resolver la ecuaci\'on
    $T(t_a)=36$.
  \item A partir del valor de $t_a$, determinar la hora del asesinato.
    \textbf{R:} A las 12:00 horas.
  \end{enumerate}
\end{enumerate}

\vfill
\hrule
\vspace{4px}
Ramiro Rebolledo Cormack\\
% Primer semestre 2022
\'Ultima edici\'on de este documento: \today

\end{document}
