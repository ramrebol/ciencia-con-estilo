\documentclass[11pt]{article}
\usepackage[spanish]{babel}
\usepackage{fancybox}

\usepackage{geometry}
\geometry{
  letterpaper,
  total={216mm,279mm},
  left=20mm,
  right=20mm,
  top=20mm,
  bottom=40mm
}

% \usepackage{graphicx} 

\usepackage{amsmath,amssymb,amsfonts,amssymb}                          
\usepackage{dsfont}
\usepackage{multicol}
\usepackage{xcolor}
\usepackage{enumerate}


\parindent=0cm                                  

\newcommand{\ds}{\displaystyle}

\newcommand{\RR}{\mathbb{R}}
\newcommand{\ZZ}{\mathbb{Z}}
\newcommand{\NN}{\mathbb{N}}

\everymath{\displaystyle}

\begin{document}
\thispagestyle{empty}
%%%%%%%%%%%%%%%%%%%%%%%%%%%%%%%%%%%%%%%%%%%%%%%%%%%%%%%%%%%%%%%%%%%%%%%%%%%%%%%%
UNIVERSIDAD DE CONCEPCI\'ON
\\
\textbf{\small FACULTAD DE INGENIER\'IA AGR\'ICOLA}
\\
DEPARTAMENTO DE AGROINDUSTRIAS\\
\rule{16cm}{.5pt}
%  \renewcommand{\theenumi}{\Roman{enumi}}
% \renewcommand{\theenumii}{\arabic{enumi}.\arabic{enumii}}
% 
% \renewcommand{\theenumiii}{\arabic{enumi}.\arabic{enumii}.\arabic{enumiii}}

\vspace{0.5cm}

\begin{center}
  \textbf{Listado 6: Integrales impropias}
  \\
  \textsc{C\'alculo II}
\end{center}
%%%%%%%%%%%%%%%%%%%%%%%%%%%%%%%%%%%%%%%%%%%%%%%%%%%%%%%%%%%%%%%%%%%%%%%%%%%%%%%%

\begin{enumerate}

\item Estudie la convergencia de las siguientes integrales:
  \begin{multicols}{2}
    \begin{enumerate}
    \item \textcolor{red}{HACER}
    \item  $\int_{1}^{2} \frac{1}{x \ln(x)} \,dx$
    \item {\bf (P)}$\int_{3}^{4} \frac{1}{(x-4)^2} \,dx$
    \item $\int_{0}^{1} \frac{1}{(1-x)^2} \,dx$
    \end{enumerate}
  \end{multicols}
  \textcolor{red}{Haga que los items que aparecen en el Problema 1 se
  visualicen en s\'olo una fila.}

  %%%%%%%%%%%%%%%%%%%%%%%%%%%%%%%%%%%%%%%%%%%%%%%%%%%%%% 

\item Usando el Teorema Criterio de Comparaci\'on directa, estudie la
  convergencia de las siguientes integrales:
  \begin{multicols}{2}
    \begin{enumerate}
    \item $\int_{1}^{5}\frac{1}{\sqrt{x^4-1}}\, dx$
    \item {\bf (P)} $\int_{3}^{6} \frac{\ln(x)}{(x-3)^4} \,dx$
    \item {\bf (P)} $\int_{1}^{\infty}\frac{1}{e^{x}+x^{2}}\,dx$
    \item $\int_{1}^{\infty} \frac{1}{xe^{x}} \,dx$
    \item $\int_{1}^{\infty}\frac{1}{\sqrt{x^{3}+1}}\,dx$
    \item \textcolor{red}{HACER}
    \end{enumerate}
  \end{multicols}

  % ----------------------------------------------------------------------------

\item Usando el Teorema Criterio de Comparaci\'on al L\'imite, estudie la
  convergencia de las siguientes integrales:
  \begin{multicols}{4}
    \begin{enumerate}
    \item {\bf (P)}$\int_{1}^{\infty}\frac{x+1}{x^3+4}\, dx$
    \item $\int_{0}^{\pi/4}\frac{\sin(x)}{x^{2}}dx$
    \item $\int_{1}^{\infty}\frac{3}{e^{x}+5}dx$
    \item $\int_{1}^{\infty}\frac{1}{e^{x}+1}\, dx$ 
    \end{enumerate}
  \end{multicols}

\item Muchas integrales se pueden calcular haciendo uso de la Funci\'on Gamma.
  Por ejemplo, se puede probar que:
  \begin{itemize}
  \item \textcolor{red}{HACER}
  \item $\int_{0}^{\infty}\frac{x^{p-1}}{1+x}\,dx
    =\Gamma(p)\Gamma(1-p)=\frac{\pi}{\sin(p\pi)}$, $ \quad 0<p<1$
  \end{itemize}

  Calcule el valor de las siguientes integrales usando la Funci\'on Gamma:
  \begin{multicols}{3}
    \begin{enumerate}
    \item {\bf (P)}$\int_{0}^{\pi/2}\sin^{6}(\theta)d\theta$
    \item $\int_{0}^{\pi/2}\sin^{4}(\theta)\cos^{5}(\theta)d\theta$
    \item $\int_{0}^{\infty}\frac{1}{1+x^{4}}dx$
    \end{enumerate}
  \end{multicols}
  \textcolor{red}{Defina un nuevo comando para que aparezca
    $\operatorname{\sen}$ en vez de $\sin$.}


  % 
\end{enumerate}


\vfill
\hrule
\vspace{4px}
\noindent RRC. \\
\'Ultima edici\'on de este documento: \today

\end{document}