\documentclass[11pt]{article}
\usepackage[spanish]{babel}
\usepackage{amsmath,amssymb,amsfonts,mathtools}

\usepackage{geometry}
\geometry{
  letterpaper,
  total={216mm,279mm},
  left=20mm,
  right=20mm,
  top=20mm,
  bottom=40mm
}

\usepackage{xcolor}
\usepackage{graphicx}
\usepackage{multicol}
\usepackage{hyperref}

\parindent=0pc

\newcommand{\ds}{\displaystyle}

\newcommand{\RR}{\mathbb{R}}
\newcommand{\ZZ}{\mathbb{Z}}
\newcommand{\NN}{\mathbb{N}}

\newcommand{\ihat}{\hat{\imath}}
\newcommand{\jhat}{\hat{\jmath}}
\newcommand{\khat}{\hat{k}}


%\newcommand{\ihat}{\boldsymbol{\hat{\textbf{\i}}}}
%\newcommand{\jhat}{\boldsymbol{\hat{\textbf{\j}}}}

\begin{document}
\noindent

UNIVERSIDAD DE CONCEPCI\'ON
\\
\textbf{\small FACULTAD DE INGENIER\'IA AGR\'ICOLA}
\\
DEPARTAMENTO DE AGROINDUSTRIAS\\
\rule{16cm}{.5pt}
% \renewcommand{\theenumi}{\Roman{enumi}}
% \renewcommand{\theenumii}{\arabic{enumi}.\arabic{enumii}}
% 
% \renewcommand{\theenumiii}{\arabic{enumi}.\arabic{enumii}.\arabic{enumiii}}
% {\usecounter{enumi}}
% \boldmath

\vspace{0.5cm}

\begin{center}
  \textbf{Listado 3: Sistemas de ecuaciones}
  \\
  \textsc{\'Algebra Lineal}
\end{center}

% ------------------------------------------------------------------------------

\begin{enumerate}

\item Escriba los siguientes sistemas de ecuaciones en
  forma matricial y resu\'elvalos utilizando el m\'etodo de eliminaci\'on
  de Gauss.
  \begin{multicols}{2}
    \begin{enumerate}
    \item
      $\left\{
        \begin{array}{rcl}
          -x_{1}+x_{3}+2x_{4}        & =&1\\
          x_{1}-x_{2}+x_{3}-2x_{4}     &  =&-1\\
          -x_{1}-2x_{2}+x_{3}+3x_{4}   &  =&2\\
          -x_{1}-4x_{2}+2x_{3}+4x_{4}  &  =&5
        \end{array}
      \right.
      $
    \item
      \textcolor{red}{COMPLETAR}
    \end{enumerate}
  \end{multicols}

  % ----------------------------------------------------------------------------
  
\item Considere $A$ y $\vec{b}$, matrices y vector de coeficiente reales,
  respectivamente, definidos por
  \[
    A=
    \begin{pmatrix}
      -a & 2 & 0 & 1\\
      a & -3 & 2 & -1\\
      a & -2 & -1 & 1\\
      2a & -2 & -4 & b
    \end{pmatrix}
    \quad
    \text{ y }
    \quad
    \vec{b}=
    \begin{pmatrix}
      1\\
      -2\\
      -1\\
      a+b+2
    \end{pmatrix}
  \]
  
  \begin{enumerate}
  \item Determine los valores de $a$ y$b$ para los cuales el sistema
    $A\vec{x}=b$, con $\vec{x}\in\RR^3$, sea compatible determinado,
    compatible indeterminado, o indeterminado.
    
  \item Para $a=1$ y $b=-1$, encuentre las soluciones del sistema.
  \end{enumerate}
  
  % ----------------------------------------------------------------------------
  
\item Considere el sistema de ecuaciones%
  \\ \\
  
  \textcolor{red}{COMPLETAR}
  \\ \\
  
  Encuentre el o los valores de $a\in\RR$ tales que el sistema sea
  compatible determinado. En cada uno de los casos, encuentre la soluci\'on
  de este sistema.

\item Sean $\alpha,\beta\in\RR$.
  Considere el sistema de ecuaciones de incógnitas $x,y,z\in\RR$
  % 
  \begin{align*}
    x+y+\alpha z  &  =0\\
    x+y+\beta z  &  =0\\
    \alpha x+\beta y+z  &  =0
  \end{align*}
  
  
  \begin{enumerate}
  \item Determine $\alpha$ y $\beta$ tales que la soluci\'on de este sistema de
    ecuaciones sea \'unica. ?`Cu\'al es dicha soluci\'on?
    
  \item Determine $\alpha$ y $\beta$ tales que el sistema tenga soluci\'on no
    trivial, y su respectivo conjunto soluci\'on.
  \end{enumerate}
  
  
  
\item Encuentre los valores de $\alpha,\beta,k\in\RR$ para que
  el sistema de ecuaciones
  \\ \\
  
  \textcolor{red}{COMPLETAR}
          
  \begin{enumerate}
  \item Tenga soluci\'on \'unica
    
  \item Sea incompatible
  \end{enumerate}
  
  
\item Sean $\alpha,\beta\in\RR$. Considere el siguiente sistema de
  ecuaciones.
  \\ \\
  \textcolor{red}{COMPLETAR}
  
  
  
  
  \begin{enumerate}
  \item Determine el o los valores de $\alpha$ y $\beta$ tales que este sistema
    de ecuaciones sea compatible determinado, compatible indeterminado
    o incompatible
    
  \item Para $\alpha=1$ y $\beta=2$, determine el conjunto soluci\'on.
  \end{enumerate}
  
\item Considere los puntos $A=(-1,0,1)$,
  $B=(0,1,2)$
  y
  $C=(1,1,1)$, y los vectores
  $\vec{a}=\overrightarrow{OA}$,
  $\vec{b}=\overrightarrow{OA}$ y
  $\vec{c}=\overrightarrow{OC}$,
  donde $O=(0,0,0)$.

  \begin{enumerate}
  \item Calcule $d(A,B)  $, $d(A,C)$,
    $d(B,C)  $, $\|\vec{a}\| $, $\|\vec{b}\|$, $\|\vec{c}\|$,
    $\left(  2\vec{a}\right)
    \cdot\vec{b}$, $\left(  -\vec{b}\right)  \times\vec{c}$
    y $\vec{b}\cdot\left(\vec{c}\times\vec{a}\right)  $.
    
  \item Determine los \'angulos entre $\vec{a}$, $\vec{b}$ y $\vec{c}$.
    
  \item Describa el conjunto de los vectores $\vec{x}\in\RR^3$
    tales que
    %
    \begin{enumerate}
    \item $\vec{x}$ es paralelo a $\vec{c}$ y $\|\vec{x}\|=1$.
    \item $\vec{x}$ es perpendicular a $\vec{a}$ y $\vec{c}$.
    \item $\|\vec{x}-\vec{a}\| =\|\vec{x}-\vec{b}\|$.
    \end{enumerate}
    
  \end{enumerate}
  
\item Determine un vector $\vec{r}\in\RR^3$ tal que
  %
  \begin{enumerate}
  \item $\left\Vert \vec{r}\right\Vert =4$, el \'angulo entre $\vec{r}$ e
    $\ihat$ es de $\dfrac{\pi}{4}$ y el \'angulo entre $\vec{r}$
    y $\jhat$ es de $\dfrac{\pi}{3}$.
    
  \item $\vec{r}$ es perpendicular a $\left[  1,2,2\right]  $,
    el \'angulo entre $\vec{r}$ e $\ihat$ es de
    $\dfrac{\pi}{6}$ y el \'angulo entre $\vec{r}$ y $\khat$
    es de $\pi$ .
    
  \item $\vec{r}$ es paralelo a $\left[  0,-1,2\right]  $, el \'angulo entre
    $\vec{r}$ e $\ihat$ es de $\dfrac{\pi}{2}$ y el \'angulo
    entre $\vec{r}$ y $\khat$ es de $\pi$ .
  \end{enumerate}
  
\end{enumerate}


\vfill
\hrule
\vspace{4px}
Ramiro Rebolledo Cormack\\
\'Ultima edici\'on de este documento: \today
        
\end{document}
